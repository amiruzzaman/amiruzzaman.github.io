\documentclass{beamer}

\usepackage{listings}
\usepackage{xcolor} % For syntax highlighting colors
%\usepackage[backend=biber]{biblatex} % For citations and bibliography
\usepackage[natbib=true,style=authoryear,backend=bibtex,useprefix=true]{biblatex}
\addbibresource{ref.bib} % Load the bibliography file

% Listings configuration for C language
\lstset{
  language=C,
  basicstyle=\ttfamily\footnotesize,
  keywordstyle=\color{blue},
  stringstyle=\color{red},
  commentstyle=\color{green!50!black},
  showstringspaces=false,
  frame=single,
  breaklines=true
}

\title{C Programming: printf Format Specifiers}
\author{Dr. Amir}
%\date{\today}
\date{}

\begin{document}

% Title slide
\begin{frame}
  \titlepage
\end{frame}

% Introduction slide
\begin{frame}{Introduction}
  \begin{itemize}
    \item The `printf` function in C is used for formatted output \citep{kernighan1988c}.
    \item Format specifiers control how variables are printed.
    \item Common specifiers:
    \begin{itemize}
      \item `\%d` – Integer
      \item `\%f` – Floating-point
      \item `\%c` – Character
      \item `\%s` – String
      \item `\%o` – Octal
      \item `\%x` – Hexadecimal
    \end{itemize}
  \end{itemize}
\end{frame}

% Slide for Integer Format Specifier (%d)
\begin{frame}[fragile]{Integer Format Specifier: \texttt{\%d}}
  \begin{itemize}
    \item Used to print integers.
    \item Example:
  \end{itemize}
  \begin{lstlisting}
#include <stdio.h>

int main() {
  int a = 10;
  printf("Integer: %d\n", a);
  return 0;
}
  \end{lstlisting}
  \textbf{Output:} \texttt{Integer: 10}
\vskip 0.5cm
  \footnotesize{This example is adapted from ( \cite{prata2004c}).}
\end{frame}

% Slide for Floating-Point Format Specifier (%f)
\begin{frame}[fragile]{Floating-Point Format Specifier: \texttt{\%f}}
  \begin{itemize}
    \item Used to print floating-point numbers.
    \item Example:
  \end{itemize}
  \begin{lstlisting}
#include <stdio.h>

int main() {
  float b = 3.14159;
  printf("Float: %.2f\n", b);
  return 0;
}
  \end{lstlisting}
  \textbf{Output:} \texttt{Float: 3.14}
\end{frame}

% Slide for Character Format Specifier (%c)
\begin{frame}[fragile]{Character Format Specifier: \texttt{\%c}}
  \begin{itemize}
    \item Used to print a single character.
    \item Example:
  \end{itemize}
  \begin{lstlisting}
#include <stdio.h>

int main() {
  char c = 'A';
  printf("Character: %c\n", c);
  return 0;
}
  \end{lstlisting}
  \textbf{Output:} \texttt{Character: A}
\end{frame}

% Slide for String Format Specifier (%s)
\begin{frame}[fragile]{String Format Specifier: \texttt{\%s}}
  \begin{itemize}
    \item Used to print strings.
    \item Example:
  \end{itemize}
  \begin{lstlisting}
#include <stdio.h>

int main() {
  char str[] = "Hello, Class!";
  printf("String: %s\n", str);
  return 0;
}
  \end{lstlisting}
  \textbf{Output:} \texttt{String: Hello, Class!}
\end{frame}

% Slide for Octal Format Specifier (%o)
\begin{frame}[fragile]{Octal Format Specifier: \texttt{\%o}}
  \begin{itemize}
    \item Used to print integers in octal (base 8) format.
    \item Example:
  \end{itemize}
  \begin{lstlisting}
#include <stdio.h>

int main() {
  int a = 10;
  printf("Octal: %o\n", a);
  return 0;
}
  \end{lstlisting}
  \textbf{Output:} \texttt{Octal: 12}
\end{frame}

% Slide for Hexadecimal Format Specifier (%x)
\begin{frame}[fragile]{Hexadecimal Format Specifier: \texttt{\%x}}
  \begin{itemize}
    \item Used to print integers in hexadecimal (base 16) format.
    \item Example:
  \end{itemize}
  \begin{lstlisting}
#include <stdio.h>

int main() {
  int a = 255;
  printf("Hexadecimal: %x\n", a);
  return 0;
}
  \end{lstlisting}
  \textbf{Output:} \texttt{Hexadecimal: ff}
\end{frame}

\begin{frame}{Step-by-Step Conversion: 255 to Hex}
\begin{enumerate}
    \item Divide 255 by 16: \quad $255 \div 16 = 15$ remainder $15$
    \item Convert remainder $15$ to hexadecimal: \textbf{f}
    \item Divide quotient 15 by 16: \quad $15 \div 16 = 0$ remainder $15$
    \item Convert remainder $15$ to hexadecimal: \textbf{f}
    \item Combine remainders in reverse order: \textbf{ff}
\end{enumerate}
\end{frame}

% Conclusion slide
\begin{frame}{Conclusion}
  \begin{itemize}
    \item Format specifiers are crucial for formatted output in C.
    \item Use them correctly to display different data types \citep{kernighan1988c, knuth1973art}.
    \item Explore more specifiers like \texttt{\%u} (unsigned integer) and \texttt{\%e} (scientific notation).
  \end{itemize}
\end{frame}


% Unsigned Integer (%u) example
\begin{frame}[fragile]{Unsigned Integer Format Specifier: \texttt{\%u}}
  \begin{itemize}
    \item Used to print unsigned integers (non-negative values).
    \item Example:
  \end{itemize}
  \begin{lstlisting}
#include <stdio.h>

int main() {
  unsigned int a = 4294967295;
  printf("Unsigned Integer: %u\n", a);
  return 0;
}
  \end{lstlisting}
  \textbf{Output:} \texttt{Unsigned Integer: 4294967295}
  \vskip 0.5cm
  \footnotesize{This example illustrates how unsigned integers are represented \citep{prata2004c}.}
\end{frame}

% Scientific Notation (%e) example
\begin{frame}[fragile]{Scientific Notation Format Specifier: \texttt{\%e}}
  \begin{itemize}
    \item Used to print floating-point numbers in scientific notation (exponential format).
    \item Example:
  \end{itemize}
  \begin{lstlisting}
#include <stdio.h>

int main() {
  double b = 1234567.89;
  printf("Scientific Notation: %e\n", b);
  return 0;
}
  \end{lstlisting}
  \textbf{Output:} \texttt{Scientific Notation: 1.234568e+06}
  \vskip 0.5cm
  \footnotesize{Scientific notation is commonly used for very large or very small numbers \citep{knuth1973art}.}
\end{frame}


% Bibliography slide
\begin{frame}[allowframebreaks]{References}
  \printbibliography
\end{frame}

\end{document}
